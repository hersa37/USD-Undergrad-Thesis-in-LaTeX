%======================== USD Thesis Template ==============================%
% Made according to the guide, keeping in mind the guide was made for Word
% TODO: Make a copy of this file that can be used right away once everything is done
\documentclass[oneside,12pt]{book}


\usepackage{titlesec} % Manage chapter and sections
\usepackage{fontspec} % Manage fonts
\usepackage{fancyhdr} % Manage header styles
\usepackage{lipsum} % Lorem Ipsum
\usepackage{graphicx} % to add images
\usepackage[portrait, left=4cm, right=3cm, top=3cm, bottom=3cm]{geometry} % set margins
\usepackage{indentfirst} % indents first line after section
\usepackage{setspace}
\usepackage{listings} % Allow writing code
\usepackage{courier} % Code listing font family
\usepackage{import} % used to import files from different directories
\usepackage{float} % Adds more image float positions
\usepackage{pythontex} % Because
\usepackage[numbib]{tocbibind}
\usepackage[labelfont=bf, labelsep=period]{caption} % Manages captions. Changed to bold and separated using period
\usepackage{pdfpages} % Allows adding pdfs as pages
% \usepackage{tocloft} % TOC modifications
\usepackage{caption} % Used to create custom caption groups
\usepackage{amsmath} % Package for math
\usepackage{scrwfile} % used to edit the appendix toc
\usepackage{hyperref} % Hyperlink for references
\usepackage{appendix}
\usepackage{pythontex}
\usepackage{pgfplots} % For plotting graphs



%============================ Title details =================================%
\def\title{An Example of Skripsi Made Using \LaTeX}
\def\author{Echa}
\def\nim{080989999}
\def\prodi{Informatika}
\def\fakultas{Sains dan Teknologi}
\def\sarjana{Komputer}


%============================= Document Imports =============================%

%============================= Citations ====================================%
% Needs to make sure why \parencite doesn't return an author-year format
% 	Solved. No need to use \parencite*, use regular \parencite
\usepackage[
	style=apa
	]
	{biblatex}

\addbibresource{examplecitations.bib}
\addbibresource{Skripsi/citations.bib}
\addbibresource{Skripsi/websiteCitations.bib}






%============================ Fonts ======================================%
\setmainfont{Times New Roman}


%========================== Page Numbering =================================%
% Use fancyhdr styles
\pagestyle{fancy}

% Header settings
\fancyhead[L, C]{} % Resets the left and center header to show nothing
\renewcommand{\headrulewidth}{0pt} % removes the header line
\fancyhead[R]{\textbf{Example Header}} % Shows text on right header

% Footer settings
\fancyfoot{} % Clears settings
\fancyfoot[L]{Made by Hersa} % shows up even if page doesn't have numbering
\fancyfoot[R]{\thepage}

% alter chapter-page numbering
\fancypagestyle{plain}{
	\fancyhf{} % resets page numbering for plain style
	\fancyhead[R]{\thepage} % sets page number to the top right of the page

}


%=================== Chapter numbering =====================%
\renewcommand{\chaptermark}[1]{\markboth{#1}{}}

\titleformat % used to edit titles with titlesec
	{\chapter} % which command to edit
	[display] % shape
	{\bfseries\Large\centering} % format of the title
	{Bab \ \thechapter} % label
	{0.5ex} % seperator
	{
		\centering
	} % before-code
	[
	\vspace{-0.5ex}%
	] % after-code

\titlespacing{\chapter}{0pt}{0pt}{20pt} % Adjusts chapter title margin


%================== Quote modifications =============%
\renewenvironment{quote}{% Quotes are internally lists, somehow. Probably unmarked list
	\list{}{%
		\leftmargin0.5in   % this is the adjusting margin
		\rightmargin0cm
	}
	\item\relax
}
{\endlist}

%============= Table of contents modifications ==========%
% TODO: - Add chapter label to table of contents.
% TODO: - Make TOC entries uppercase and not bold.
% TODO: - Change filler to be more dots
\renewcommand{\contentsname}{Daftar Isi}
\renewcommand{\listfigurename}{Daftar Gambar}
\renewcommand{\lstlistlistingname}{Daftar Listing}
\renewcommand{\listtablename}{Daftar Tabel}

% TODO: Figure out how to make this work with only main matter
% TODO: Figure out why using tocloft made TOC into sections instead of chapters
%\renewcommand{\cftchappresnum}{Chapter ~}
%\renewcommand{\cftfigpresnum}{figure~}
%\renewcommand{\cfttabpresnum}{table~}
%\renewcommand{\cftchapnumwidth}{3cm}
%\renewcommand{\cftfignumwidth}{2cm}
%\renewcommand{\cfttabnumwidth}{2cm}
%\addtocontents{toc}{\protect\contentsline{chapter}{Chapter:}{Page}}
%\usepackage{capt-of}

% TODO: HOW DO I MAKE APPENDICES WORK?????
\TOCclone[Daftar Lampiran]{toc}{atoc}
\addtocontents{atoc}{\protect\value{tocdepth}=-1}
\newcommand\listofappendices{\listofatoc}

\newcommand*\savedtocdepth{}
\AtBeginDocument{%
	\edef\savedtocdepth{\the\value{tocdepth}}%
}



%=================== Appendix modifications ===============%

%================== Graph Modifications
\pgfplotsset{width=10cm, compat=1.9}

\usepgfplotslibrary{external}



%============= Caption label modifications ==============%
\renewcommand{\figurename}{Gambar}

%================== Code block modifications =============%
% TODO: Maybe find a way to add colors
\lstset{numbers=left, numberstyle=\tiny}
\lstdefinestyle{skripsilisting}{numbers=left, numberstyle=\tiny, lineskip=-0.8ex, basicstyle=\small}

%================ Start of document ==================%
\begin{document}

%================== Custom title page ================%
% TODO: Check if title page is correct or not
\begin{titlepage}
	\begin{center}
		\large
		\textbf{\MakeUppercase{\title}}
		\vspace{2ex}


		\textbf{SKRIPSI}

		\vspace{2ex}

		Diajukan untuk memenuhi salah satu syarat memeroleh gelar Sarjana \sarjana \\Program Studi \prodi


		\vspace{3cm}
		\includegraphics[width=5cm]{usd}\\
		\vspace{1.5cm}
		Disusun oleh:\\
		\author\\
		NIM: \nim\\

		\vspace{2cm}

		\MakeUppercase{
			Fakultas \fakultas\\
			Universitas Sanata Dharma\\
			Indonesia\\
			\the\year{}\\
		}
	\end{center}
\end{titlepage}

%=================== Front matters =======================%
% Bastracts, page of thanks, table of content/figures/etc
\frontmatter

% uses \addcontentsline to add PDF into TOC. Should add the line before adding the PDF so it's on the same page as the start of PDF
\addcontentsline{toc}{chapter}{Inserted PDF}
\includepdf{Breakwater.pdf}


\chapter{Kata Pengantar}

\chapter{Abstrak}

\chapter{Abstract}

\tableofcontents

\listoffigures

\listoftables

\lstlistoflistings

\listofappendices





%=================== Main part of document =================%
% Chapters etc
% TODO: Check if spacing is correct
\mainmatter

\begin{doublespace}

% \spacing{1.213}
% Pendahuluan
\subimport{Skripsi/Pendahuluan}{pendahuluan}

% Landasan Teori
\subimport{Skripsi/Landasan_Teori}{landasan_teori}

\chapter{Trial}
\section{Test}
\subsection{A subsection}
\lipsum[1-2]

\begin{pycode}
print("hello world")
\end{pycode}
% manual code listing

\begin{lstlisting}[caption=Manual input, label={listing-java-manual},language=java,style=skripsilisting]
public class HelloWorld {
	public static void main(String[] args) {
		System.out.println("Hello world");
	}
}
\end{lstlisting}

\lstinputlisting[caption=Test java, label={lst:listing-java}, language=java, style=skripsilisting]{Mahasiswa.java}

\chapter{Landasan Teori}
\lipsum[1-2]

\section{Graphs}
\begin{tikzpicture}
	\begin{axis}[
		axis lines = center,
		xlabel = \(x\),
		ylabel = {\(f(x)\)},
		]
		%Below the red parabola is defined
		\addplot [
		domain=-10:10,
		samples=100,
		color=red,
		]
		{x + 1};
		\addlegendentry{\(x + 1\)}
		%Here the blue parabola is defined
		\addplot [
		domain=-10:10,
		samples=100,
		color=blue,
		]
		{-x};
		\addlegendentry{\(-x\)}

	\end{axis}
\end{tikzpicture}

\section{Block quotes}
This part is an example of making a quote. We'll fill it out with stuff so it becomes more than one line without using lorem ipsum
\begin{quote}
	\lipsum[1-2] \parencite{wombat2016} % Parenthesis citation.
\end{quote}

According to \footcite[Quote from][According to me, this is really neat]{lion2010}, this is something. What can we learn from this exercise? Well, if we add another paragraph, bla bla bla bla bla.

Another example of citation is using this \parencite[see][page 12]{wikibook}



\chapter{This is in a Different File}
Well well well, what do we have here?. Use this command in the document body to insert the contents of another file named filename.tex; again this file should not contain any LATEX preamble. 

\section{Math Stuff}
\lipsum[1]

\[1+1=2\]


\begin{equation}
	e=mc^2
\end{equation}



\subimport{New_Chapter}{another_one} % Imports from subdirectory


\end{doublespace}
%================= Back matters ===================%
% Bibliography and attachments



\backmatter

\printbibliography[heading=bibintoc]


\begin{appendices}
	\chapter{something}

	\section{Appendix Code}
	\begin{lstlisting}[language=java, style=skripsilisting]

		public static void main(String[] args) {
			System.out.println("Wooo, cool!");
		}
	\end{lstlisting}

\end{appendices}


\end{document}



